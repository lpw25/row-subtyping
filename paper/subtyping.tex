\documentclass{article}
\usepackage{supertabular}
\usepackage{bm}
\usepackage{mathpartir}
\usepackage{amsmath,amssymb}
\usepackage{supertabular}
\usepackage{geometry}
\usepackage{ifthen}
\usepackage{alltt}%hack
\geometry{a4paper,dvips,twoside,left=22.5mm,right=22.5mm,top=20mm,bottom=30mm}
\usepackage{color}

\input{subtyping-gen.tex}

\renewcommand{\ottusedrule}[1]{#1 \and}
\renewenvironment{ottdefnblock}[3][]
  { \framebox{ \mbox{#2} } \quad \begin{mathpar}#3 }
  { \end{mathpar} }

\title{Algebraic subtyping with invariance}
\author{Leo White}
\date{}
\begin{document}
\maketitle
\begin{abstract}
  Dolan's \emph{Algebraic subtyping} described for the first time a type
  system combining parametric polymorphism and subtyping with type
  inference, principal types and decidable type subsumption. This system
  has a restriction that type constructors must not be invariant -- they
  must be either co- or contra-variant in each type parameter. However,
  existing languages support invariant type constructors and could not
  easily remove them.

  This paper gives a new presentation of algebriac subtyping that does
  not require this restriction. Using \emph{type ranges} it allows the
  type system to have invariance. By adjusting which contexts are
  invariant, our system becomes a spectrum of type systems with
  traditional Hindley-Milner at one end (all contexts are invariant) and
  the original algebraic subtyping at the other (no contexts are
  invariant). We formalise this presentation in Coq and prove it sound
  and its type inference principal.
\end{abstract}

\section{Introduction}

\section{Language}
\subsection{Grammar}
\subsubsection{Meta-variables}
\ottmetavars
\subsubsection{Productions}
\ottgrammar

\section{Type inference}

\section{Type subsumption}

\section{Recursive Types}

\section{Simplifying type schemes}

\section{A spectrum of type systems}

\section{Related work}

\appendix
\section{Typing judgements}
\ottdefnsTyping
\section{Operational semantics}
\ottdefnsSemantics
\section{Judgements for recursive types}
\end{document}